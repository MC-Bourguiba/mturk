% Sets
\newcommand{\Fbb}{\mathbb{F}}
\newcommand{\Rbb}{\mathbb{R}}
\newcommand{\Cbb}{\mathbb{C}}
\newcommand{\Nbb}{\mathbb{N}}
\newcommand{\Qbb}{\mathbb{Q}}
\newcommand{\Zbb}{\mathbb{Z}}

\newcommand{\Acal}{\mathcal{A}}
\newcommand{\Bcal}{\mathcal{B}}
\newcommand{\Ccal}{\mathcal{C}}
\newcommand{\Dcal}{\mathcal{D}}
\newcommand{\Ecal}{\mathcal{E}}
\newcommand{\Fcal}{\mathcal{F}}
\newcommand{\Gcal}{\mathcal{G}}
\newcommand{\Hcal}{\mathcal{H}}
\newcommand{\Ical}{\mathcal{I}}
\newcommand{\Jcal}{\mathcal{J}}
\newcommand{\Kcal}{\mathcal{K}}
\newcommand{\Lcal}{\mathcal{L}}
\newcommand{\Mcal}{\mathcal{M}}
\newcommand{\Ncal}{\mathcal{N}}
\newcommand{\Ocal}{\mathcal{O}}
\newcommand{\Pcal}{\mathcal{P}}
\newcommand{\Qcal}{\mathcal{Q}}
\newcommand{\Rcal}{\mathcal{R}}
\newcommand{\Scal}{\mathcal{S}}
\newcommand{\Tcal}{\mathcal{T}}
\newcommand{\Ucal}{\mathcal{U}}
\newcommand{\Vcal}{\mathcal{V}}
\newcommand{\Wcal}{\mathcal{W}}
\newcommand{\Xcal}{\mathcal{X}}
\newcommand{\Ycal}{\mathcal{Y}}
\newcommand{\Zcal}{\mathcal{Z}}

\newcommand{\bigO}{\Ocal}

%-----------------------------------------------------------------------------------------------------------------------------------------------------------------
% operators
\DeclareMathOperator*{\argmax}{arg\,max}
\DeclareMathOperator*{\argmin}{arg\,min}
\DeclareMathOperator*{\cart}{\times}
\DeclareMathOperator*{\card}{card}
\DeclareMathOperator*{\superset}{\supset}
\DeclareMathOperator*{\support}{support}

% calculus
\newcommand \ddt[1]{\frac{d #1}{dt}}
\newcommand \ppx[1]{\frac{\partial #1}{\partial x}}
\newcommand \ppxn[2]{\frac{\partial^{#2} #1}{\partial^{#2} x}}
\newcommand \ddtn[2]{\frac{d^{#2} #1}{dt^{#2}}}

% topology
\DeclareMathOperator*{\bd}{bd}
\DeclareMathOperator*{\osc}{osc}
\DeclareMathOperator*{\disc}{disc}
\DeclareMathOperator*{\cl}{cl}
\DeclareMathOperator*{\interior}{int}

% linear albegra
\DeclareMathOperator*{\dm}{dim}
\DeclareMathOperator*{\spn}{span}
\DeclareMathOperator*{\trace}{trace}
\DeclareMathOperator*{\Tr}{Tr}
\DeclareMathOperator*{\diag}{diag}

% probability
\DeclareMathOperator*{\cov}{cov}
\DeclareMathOperator*{\var}{var}
\DeclareMathOperator*{\Exp}{\mathbb{E}}
\newcommand\Expsq[1]{\Exp\sqbr{#1}}
\DeclareMathOperator*{\Pro}{\mathbb{P}}
\DeclareMathOperator*{\Prob}{\mathbb{P}}
\newcommand \convL[1]{\overset{L^{#1}}{\rightarrow} }
\newcommand \convP{\overset{\text{P}}{\rightarrow} }
\newcommand \convAS{\overset{\text{a.s.}}{\rightarrow} }



%convex optimization
\DeclareMathOperator*{\Co}{Co}
\DeclareMathOperator*{\conv}{conv}
\DeclareMathOperator*{\diam}{diam}

%complex
\DeclareMathOperator*{\Real}{Re}
\DeclareMathOperator*{\Imag}{Im}
\newcommand\contains{\ni}



%-----------------------------------------------------------------------------------------------------------------------------------------------------------------
% text
\newcommand{\by}{\text{ by }}
\newcommand{\pf}{\paragraph{\emph{proof}}}
\newcommand\p[1]{\paragraph{#1}}
\newcommand{\ans}{\paragraph{\emph{answer}}}
\newcommand{\subjectto}{\text{subject to}}
\newcommand\ind[1]{1_{#1}}

\newcommand\emp[1]{{\color{Red} #1}}

%-----------------------------------------------------------------------------------------------------------------------------------------------------------------
% matrices and equations

\newcommand \func[5]{
\[
\begin{aligned}
#1: #2 &\rightarrow #3 \\
#4 &\mapsto #5
\end{aligned}
\]
}

\newcommand \al[1]{\begin{align*}
#1
\end{align*}
}

\newcommand \aln[1]{\begin{align}
#1
\end{align}
}

\newcommand \ald[1]{
\[
\begin{aligned}
#1
\end{aligned}
\]
}

\newcommand \aldn[1]{
\begin{equation}
\begin{aligned}
#1
\end{aligned}
\end{equation}
}


\newcommand \mat[1]{
\left(
\begin{array}
#1
\end{array}
\right)
}

\newcommand \Det[1]{
\left|
\begin{array}
#1
\end{array}
\right|
}
%-----------------------------------------------------------------------------------------------------------------------------------------------------------------
%other
\newcommand{\horline}{
\begin{center}
\line(1,0){500}
\end{center}
}

\newcommand \vs{\vspace{40pt}}

\newcommand \imp{\Rightarrow}
\newcommand \eqv{\Leftrightarrow}



%-----------------------------------------------------------------------------------------------------------------------------------------------------------------
% parenthesis and such
\newcommand \floor[1]{\lfloor #1 \rfloor}
\newcommand \ceil[1]{\left\lceil #1 \right\rceil}
\newcommand \bra{\left\langle}
\newcommand \ket{\right\rangle}
\newcommand \braket[2]{\bra #1, #2 \ket}
\newcommand{\psh}[2]{\ensuremath{\langle #1,#2\rangle}}
\newcommand \parenth[1]{\left( #1 \right)}
\newcommand \curl[1]{\left\{ #1 \right\}}
\newcommand \sqbr[1]{\left[ #1 \right]}
\newcommand \sqbra[1]{\left[ #1 \right]}


%-----------------------------------------------------------------------------------------------------------------------------------------------------------------







%-----------------------------------------------------------------------------------------------------------------------------------------------------------------
% Algorithms
\usepackage{listings} % for algorithms
\usepackage{framed}
% setup of the lst
\lstset{ %
  basicstyle=\footnotesize,
  commentstyle=\color{gray},
  extendedchars=true,
  frame=single,
  keywordstyle=\color{blue},
  language=Java,
  morekeywords={trait, def, val},
  numbers=left,
  numbersep=5pt,
  numberstyle=\tiny\color{gray},
  tabsize=2
}



