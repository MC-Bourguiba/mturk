\documentclass[10pt,a4paper]{article}

\usepackage[utf8]{inputenc}
\usepackage[english]{babel}
\usepackage{amsmath, amsfonts, amssymb, amsthm}
\usepackage{graphicx}
\usepackage{algorithm}
\usepackage{algorithmic}
\usepackage{enumerate}
\usepackage[usenames,dvipsnames]{xcolor}
\usepackage[left=1.5in,right=1.5in,top=2.5cm,bottom=2.5cm]{geometry}
\usepackage{wrapfig}
\usepackage{subcaption}

\input{latex_commands_n.tex}
\newcommand \bepsilon{{\boldsymbol \epsilon}}


% theorems
\newtheorem{definition}{Definition}[section]
\newtheorem{lemma}{Lemma}
\newtheorem{theorem}{Theorem}
\newtheorem{conjecture}{Conjecture}
\newtheorem{rem}{Remark}
\newtheorem{fact}{Fact}
\newtheorem{proposition}{Proposition}
\newtheorem{corollary}{Corollary}
\newtheorem{assumption}{Assumption}


%============================================================================================
\title{Routing games}
\date{}


%============================================================================================
\begin{document}

\maketitle

%============================================================================================

The game is given by a directed graph $G = (V, E)$, a set of edge cost functions $(c_e)_{e \in \Ecal}$, and a finite number of players, indexed by $k \in \{1, \dots, K\}$.

\begin{itemize}
\item The cost function of an edge $e$ is a function $c_e : \Rbb_+ \to \Rbb_+$. It determines the cost of the edge given the total flow on that edge.
\item A player $k$ is given by: a source node $s_k \in V$, a destination node $d_k \in V$, and a total flow $F_k$ (i.e. the total mass of traffic that this player is allocating). The action set of the player is the paths that connect $s_k$ to $d_k$, denoted by $\Pcal_k$.
\item At iteration, each player $k$ chooses a flow distribution $f_k \in \Rbb_+^{\Pcal_k}$, such that $\sum_{p \in \Pcal_k} f_{k,p} = F_k$. The flow distributions of all players determine the edge flows as follows: for an edge $e$, the edge flow is
\[
\phi_e = \sum_k \sum_{p \in \Pcal_k : e \in p} f_{k, p}
\]
Another way to write this is
\[
\phi = \sum_{k} M^k f_k
\]
where $M^k$ is an incidence matrix for player $k$, defined as follows: $M^k \in \Rbb^{\Ecal \times \Pcal_k}$, such that 
\[
M^k_{e, p} = \begin{cases}
1 & \text{if $e \in p$} \\
0 & \text{otherwise}
\end{cases}
\]
Once we have the edge flows, we can compute the edge costs simply by applying the edge cost functions. Let $y$ be the vector in $\Rbb^{E}$ defined by
\[
y_e = c_e(\phi_e)
\]
Then we compute the path costs by summing the edge costs along the path. So for all $k$, and all $p \in \Pcal_k$, the path cost is
\al{
\ell^k_p = \sum_{e \in p} y_e
}
so the path costs for player $k$ can be written simply in terms of the incidence matrix
\[
\ell^k = (M^k)^T y
\]
\end{itemize}

To summarize, when we construct the graph, we need: the node set $V$, the edge set $E$, the edge cost functions $c_e, e \in E$, and the player description $(s_k, d_k, F_k)$ for each $k$. From this, we can compute the paths $\Pcal_k$ and compute the incidence matrices $M_k$. 

Then, at each iteration, each player chooses a path flow distribution $f_k$, which we use to compute:
\begin{enumerate}
\item the edge flows $\phi$
\item the edge costs $y$
\item the path costs $\ell_k$ for each $k$
\end{enumerate}

Once this is done we reveal to each player $k$ the path cost vector $\ell_k$ and we start the next round.

\end{document}
